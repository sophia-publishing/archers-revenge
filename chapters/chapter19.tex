\chapter{}

The Minister, Guru, was near the gate of his guest house welcoming his guests
when he heard some commotion on the terrace. People were shouting and asking
others to move into the house. At least two arrows landed near the gate where he
was standing. He rushed inside the house to see many of his guests running down
into the house from the terrace.

“What happened, why are you all running down?” he asked.

“Some people in the terrace have been hit by arrows. Some one is shooting arrows
on all of us from outside. We ran down so that we don't get hurt,” one of his
guests mentioned to him while simultaneously running out of breath. Others
indicated towards the terrace and said, “Go save them. Some of them are still
struck in the terrace and battling for their lives!”

He ran to the terrace and found a few arrows scattered all around the place. By
then the terrace was almost empty except a couple who were still standing there
with glasses in their hands. “We just came back to fill up our glasses with one
last shot. The arrows have stopped now,” they said. It was difficult for him to
admire their dedication for the cause of drinking alcohol even when their lives
were in danger!

Guru came back into the house. Fortunately only one person on the terrace was
hurt and the injury was minor – the arrow head had struck his shoulders and
penetrated less than half an inch into his shoulders. “Please someone call the
ambulance. Guru, look at what happened to me - please save me,” he was wailing
frantically. Guru assured him that there was no threat to his life but to pacify
him he called the ambulance.

He canceled the party and sent his people to search for the attackers around the
area.

Guru wondered if this was another attempt to kill him. He picked up an arrow
that had landed near the gate and analyzed it. The arrowhead was blunt. The
killers did not intend to kill him or hurt anyone this time. He realized that
the intention of the attackers was probably to create panic and spoil his party.

By that time all his guests left the party “safely” in their cars “protected” by
a solid roof. The cars that had a sun-roof were sure to close it. Their life was
in danger and they were not willing to take any further chances! Guru knew the
rumor mills would have already started working overtime. By now, everyone
connected to his VIP guests would have come to know about the “shocking attack”
and how they managed to “daringly escape from the place alive”. He wondered if
Facebook and Twitter servers were equipped to handle this sudden surge in
traffic.

In a short time, as he expected, there was a flurry of reporters, camera-men,
and other members from the media. They were all lined-up in front of his house
wanting to know what had happened and who was responsible for the attack. After
all some VIPs in the city had “narrowly” managed to escape “sure” death.

In spite of his assurances that nothing serious happened and it was just an
attempt by the “opposition parties” to sabotage his party and create commotion,
the media was persistent to know the exact number of “deaths”. Their “breaking
news” wouldn't be all that “breaking” without that crucial piece of news. Guru
shut the gate tightly and asked his guards not to allow anyone inside until they
received further instructions from him. The reporters left after an hour or two
but a few of them stayed overnight determined to find out more about this
“shocking crime”. Guru had a tough time convincing the top bosses of
publications and TV channels not give this issue much coverage. He had to use
his influence and money to stop media from creating a ruckus out of a non-issue.

The last thing he wanted during the election time was negative publicity. His
security guards told him that the attackers had already left and they couldn't
locate them anywhere in the area despite a thorough search. But they had already
alerted the Police and requested them to check every vehicle going out of ECR
Highway. Guru immediately called the Police Commissioner of Crime Department and
Traffic Department and convinced them that the “attack” was actually a prank and
was coordinated by opposition members to sabotage his party. He was aware of
what was going on and they need not waste their time on a trivial issue like
this. It was difficult to convince, but he somehow “managed” to control the
issue from getting escalated.
After having taken care of the media and the police, he called the Police
Inspector of Tirupati town - Samara Simha Reddy. The inspector was about to
leave for his home when he got the call.

“Hello”

“Guru here. I gave you some leads and asked you to work on my case. Have you
done anything about it, yet?” the Minister asked.

“Yes, we are working on it. I'll inform you when we get hold of him,” the
Inspector replied.

“When will you inform me? After I am dead?” the Minister screamed, unable to
control his anger.

“Mr. Guru, please be patient. I am not your servant – you can't shout at me like
this. Be calm and tell me what happened,” the Inspector said.

“I was attacked once again, this time in my guest house in Chennai. I organized
a party for some VIPs in the city but I was forced to cancel the party as a
flurry of arrows fell from the sky. Fortunately only one person was hurt and it
was a mild injury,” the Minister replied.

“They used bows and arrows to attack you once again?” the Inspector asked.

“Yes. This time, I don't think they wanted to hurt or kill me – they used blunt
arrows. I think their intention was to spoil my party,” the Minister said.

“Why didn't you catch them? Didn't you have your people there?”

“They attacked from a distance, maybe from the beach. By the time our people
went in search of them, they had fled away. I am sure this is going to be a hot
news in the entire VIP circle of this city. I want you to do something quickly
before they manage to do more damage. Weren't you able to proceed further with
this case in someway?” the Minister asked.

“I feel, that guy Aryan, about whom you mentioned to me earlier is behind it,”
the Inspector said.

“I know that already. Who else is with him and why have you not caught him yet?”
the Minister asked.

“The problem is – he doesn't stay in the town. If he was staying in a hotel or
even a rented accommodation we would have traced him by now,” the Inspector
replied.

“Where else does he stay?” the Minister asked.

“He stays in the forest, in the hills. We are trying to locate him by tracking
his cellphone but the signal is very weak and most of the time his phone is
switched off. We need input from at least three cellular towers to locate him
accurately. But from the data we have, he can be anywhere within a fifteen
kilometer radius. That's too large an area in the forest for us to track,” the
Inspector replied.

“What do we do?” the Minister asked.

“He has to come out - our team and our associates have his photo. We have
already circled the area. Just give us time - we'll get him,” the Inspector
said.

“Please do something quickly. If you need any help, let me know. I want that guy
behind the bars soon,” the Minister said and cut the call.

Even though Guru was not satisfied with the conversation, he was confident that
this Inspector could be trusted with the job.