\chapter{}

\lettrine{A}{ryan} had two options: to go back to the UK and finish his studies, or
bring justice to his father's death, risking his own life.

Aryan couldn't imagine a life without his father—his father was everything to
him. Even though India had an active family-oriented culture, he was not
familiar with anyone in the extended family—either from his father's side or
his mother's side. His parents were shunned by both families because theirs was
a love marriage. Although he was in touch with other family members later on, he
was never close to anyone.

Aryan could feel the void created in his life due to his father's death. Despite
being in the UK, Aryan would talk to his father daily. His father was his best friend—a
guide whom he trusted deeply. Without his father, life did not mean anything to
him—he felt lifeless. He couldn't accept the fact that someone could take away
the only source of his life's support and strength.

Aryan was not the kind of person who bowed down before unjust authority. The
minister had the temerity to think that his power and money would silence
everybody. Blinded by his might, he probably thought he was invincible and could
induce fear in the mind of anyone who dared to oppose him.

But the minister did not know that the fear of death grips only those people who
value life over death. Aryan's biggest disadvantage was now his biggest
advantage—he didn't have to fear anything or anyone, any more. Aryan was ready
to sacrifice his life, but he couldn't let the minister get away with a crime as
heinous as this. Revenge seemed to be the only option for Aryan—what was the
point of living a lifeless existence anyway? The minister had not only destroyed
his father's life but also his. Aryan would not allow the minister to get away
with his murder.

Over the next few days, Aryan considered his options. Since he couldn't trust the
police any longer, he decided to take matters into his own hands. The minister seemed to
have everything—power, wealth, influence, and both the police and thugs at his disposal.
Aryan was merely an unarmed individual. Or was he?

Aryan believed that any problem could be solved if approached with dedication,
willpower, and sincerity. He did not allow himself to be deterred by the
enormity of the problem. His father taught him that any goal could be attained if
it is broken down into smaller, achievable tasks. He told him that no
challenge was insurmountable—he just had to make an honest analysis of his
strengths, weaknesses, and circumstances. If he pushed through with ardent
determination, thorough planning, and daring execution, there could be only one
result—victory.

Aryan's weaknesses in this situation were evident to him. But did he have any
strengths?

Yes.

Archery.

Aryan was a trained archer.

Archery may have lost its sheen because of the invention of gunpowder and automatic
firearms, but bows and arrows formed an important component of any war
until even a couple of centuries ago.

Aryan knew he was against power, might, and guns, but archery had its advantages,
too. The most important advantage was the surprise factor—no one would expect
him to attack with bows and arrows, and they wouldn't be prepared to defend
against this form of attack.

Archers have been an important element of many battles fought in Ancient and
Medieval India. Both Rama and Lakshmana, the heroes of the epic Ramayana, were
archers. The Mahabharata War, another famous epic, was won due to Arjuna, the
archer, and Krishna, the mentor. The commander of the enemy forces, Karna, was an
archer too. The valiance of Arjuna's son, Abhimanyu—a proficient archer—and
the sacrifice of Ekalavya—who could hit targets based solely on sound—were
epic. Archers determined the outcome of wars. Archers defended forts successfully.
Archers were central to the planning of any war.

And Aryan was an archer too.