\chapter{}

In India, people usually become successful politicians because they are the kin
of successful Ministers/MLAs, related to/friends of industrialists, or famous
movie stars/social activists.

Guru was none of them.

He was the eldest among six siblings. When he was nineteen years old, his
father, a construction laborer, died in an accident. Guru was obliged to quit
his Government College education and joined his mother in the construction work
to support his family.

Initially, he was contented with the daily wages that was provided. Since his
brothers and sisters were still studying in the Government School at Chennai,
they had access to free meals provided by the State. Money earned by Guru and
his mother was sufficient to feed the family during the nights and mornings.
But, the income of the family was not enough to get his sisters married.

In India, parents need to provide dowry (gold jewelery or cash) to get their
daughters married. The value of the dowry determined the status of the groom.
This practice was not prevalent in all communities, but in Guru's case, dowry
was very much a reality. Guru knew that he had to earn considerably more, that
too in a short period of time, if he wanted to get his sisters married into
relatively well-to-do families.

Guru noticed that some workers occasionally died or suffered major health
ailments due to poor safety standards in construction companies/sites. The
contractor whom he was employed with did not bother about the safety of workers.
Investing in the well-being of laborers did not seem like a good investment to
the contractor. Guru noticed that there was no labor union to fight for the
rights of laborers. He formed an informal labor union and started negotiating
with his employer on behalf of the affected families. He tried his best for
monetary compensation to the families of laborers who died or suffered serious
ailments as a result of their strenuous working conditions.

Initially, he faced a lot of opposition from his employer for his attempts to
bring justice to the laborers. He was attacked by thugs hired by his employer,
but he managed to survive the assault. The incident only increased his
determination to fight. His efforts were noticed by other laborers and he formed
a network of able-bodied laborers in order to fight any such attacks in the
future. Unable to take him down by force, his employer relented and offered
monetary compensation for the affected families. It turned Guru a hero among the
laborers and he became their savior. Laborers now consulted their problems
concerned with company management and he became their unelected representative.

However, Guru found that he was unable to make any money out of these endeavors.
What he did amounted to social service, not business. Since he had secured the
trust of laborers, Guru decided that it was time to cash on that trust. He now
negotiated with both parties to strike the best deal possible – for the
employer. He was paid a handsome commission for these cost-reduction efforts.
Guru found that he was not only able to make a decent amount of money, but was
also much sought after and thanked by both laborers and the employer. Even
though he didn't realize it, he was already learning the art of politics.

“Why should I ask my workers to vote for you? What have you done for our company
over the last few years as a Ward Councilor?” Guru heard his boss fuming over
the phone as he entered the General Manager's cabin. The boss said, “Look at
this Guru – This useless ward councilor won the previous election because I
recommended his name to all our laborers. But, he conveniently forgot about us
after that. Now he wants me to canvass for him again. What makes him think that
I will agree to it is beyond my comprehension. Anyway, what brings you here?”

Guru said, “You called me today morning to discuss about the compensation to be
given to-”

His boss interrupted. “Oh, that. You do what best you can. I know you'll have
the company's interest as your topmost priority. Am I right?”

Guru said, “Sir, I have always had the company's interest as my topmost
priority. You know that.”

His boss nodded his head and got into thoughtful contemplation for a few
seconds. He then looked at Guru and said, “Guru, I have an idea – what stops you
from contesting the Ward Councilor election yourself? I am sure all our workers
will support you. Since you live in this area and you know almost everybody in
the colony, you should be able to convince people to vote for you. You have a
good name too. What do you think?”

Guru said, “But I have never contested an election before, Sir. Besides, I don't
have the money required for campaigning.”

His boss said, “Don't worry about funds. I'll sponsor your campaign. It's time
you stop slogging as a laborer and start handling bigger responsibilities. A
journey of thousand kilometers starts with a single step. Go ahead and file your
nomination papers - you have my blessings.”

Guru didn't anticipate how big that responsibility might get, one day. It was
then Guru visited the Tirupati Temple for the first time. A day before the
election results were announced, he turned nervous and someone suggested him to
visit the powerful God of the seven hills in Tirupati. To get over his anxiety,
Guru boarded the immediate bus to Tirupati, even though he did not have the
habit of visiting Temples.

As soon as he reached Tirupati, he walked to Alipiri and then walked all the way
up by foot instead of taking another bus to reach the hilltop Tirumala Temple.
He reached the Temple at midnight and he had to wait for twelve hours to have
the darshan of the God. Even though this journey was physically exhausting, it
soothed his nerves and he got rid of all the mental tension he had on the
previous day. After coming out of the Temple, Guru was much more relaxed. He
knew that the result would have been announced by then. He was tempted to call
home, but instead he bought food, distributed it among the hungry poor, ate some
food and boarded a direct bus to Chennai.

He reached Chennai at evening and went home directly. He knocked the door, his
wife rushed to open it, beamed, and informed him that he won the elections. He
embraced her, looked into her eyes and said calmly, “I know”.

There after, he made it a point to visit the Tirupati Temple at least once a
year. Thirty five years passed since then. From being a Ward Councilor, Guru
become M.L.A - Member of Legislative Assembly (State Minister), and later on
Member of Parliament (Central Minister). During his second term as a Central
Minister he managed to wrestle the 'plum' Energy/Power portfolio from a rival.
The rival Minister, however, convinced the Prime Minister to split the Energy
portfolio into two and became in-charge of the Renewable Energy portfolio.

This was a major setback to Guru as new power projects were about to be
sanctioned to produce power using Renewable Energy technologies, and additional
funds were released to subsidize the rooftop and commercial solar installations.
The renewable energy projects were completed very quickly when compared to the
fossil fuel and nuclear-based power plants he was setting up. There were plenty of
business opportunities in the Renewable Energy Ministry and all of them were
snatched away from him.

Guru did not take this defeat lightly. Not only was he addicted to money and
power, but he also wanted to be invincible. For the first time, he chose not to
circumvent roadblocks in his path – he wanted to uproot them, reduce them into a
rubble and trample triumphantly over all that dust. That, he thought, would be a
fitting reply to his rival.
